\documentclass[UTF8]{ctexart}



\begin{document}

\documentclass[12pt, a4paper, oneside]{ctexart}
\usepackage{amsmath, amsthm, amssymb, appendix, bm, graphicx, hyperref, mathrsfs,abstract}

\title{\textbf{关于基因组与生物进化}}
\author{QI BIN \ \ \ 211840108}
\date{\today}
\linespread{1.5}

% This will generate a table of contents
\renewcommand{\abstractname}{\Large\textbf{摘要}}
\maketitle

\setcounter{page}{0}
\maketitle
\thispagestyle{empty}

\begin{abstract}
传统的生物进化,系统发育,及生物分类的依据主要是解剖学和生物形态学证据。现在,随着微观生物学的发展, 人们对遗传物质,即DNA,和基因组的发展,以及高通量测序技术的普及,人们对生物进化与分类有了更深的理解。
    
\end{abstract}

\newpage
生物进化和系统发育一直是生命科学的一个大论题,这个论题可以归结为“生命从哪里来,生物如何发展成现在这个样子,地球上那么多种生物该如何分类,生物之间又有什么进化关系”。对于这一问题的解释也多种多样,我今天只挑几个有意思的话题展开讨论。\par
我们先聊一聊生物进化的方向。\par 初中课本中写道,生物体的进化方向是从水生到陆生,从简单到复杂,从低等到高等。这一假设虽然可以以化石证据,即在深的地层中没有找到高等生物的化石,加以佐证,但是一些现存的生物可以反驳这一观点。比如鲸类是一种水生的哺乳动物,盲鱼在发育过程中失去了视觉。这都与传统的理论相反\par 所以科学家们提出了新的理论,现在普遍认可的理论是进化由基因突变的积累造成,是完全随机的过程,自然选择使有利的突变被保留,不利的突变被淘汰,为进化提供方向,同时造就适应,生物就在这样的选择压力下形态发生改变,越来越适应当前的环境。这一理论至今未被推翻,且可以完美解释目前的现象。还以鲸类举例。对基因序列的分析表明,鲸与牛等食草动物亲缘关系较近,同属鲸偶蹄目。可以认为,鲸的祖先是一种小型陆生食草动物,在沧龙灭绝后生态位空闲,碰巧此时古鲸在水边活动,一些适应水生生活的基因(性状)被保留下来,于是下水,占据了海洋大型捕食者的生态位,逐渐适应水生生活与食肉。




\end{document}