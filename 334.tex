\documentclass[UTF8]{ctexart}



\begin{document}

\documentclass[12pt, a4paper, oneside]{ctexart}
\usepackage{amsmath, amsthm, amssymb, appendix, bm, graphicx, hyperref, mathrsfs,abstract}

\title{\textbf{关于基因组与生物进化}}
\author{QI BIN \ \ \ 211840108}
\date{\today}
\linespread{1.5}

% This will generate a table of contents
\renewcommand{\abstractname}{\Large\textbf{摘要}}
\maketitle

\setcounter{page}{0}
\maketitle
\thispagestyle{empty}

\begin{abstract}
传统的生物进化,系统发育,及生物分类的依据主要是解剖学和生物形态学证据。现在,随着微观生物学的发展, 人们对遗传物质,即DNA,和基因组的认识更加深入,,以及高通量测序技术的的发展与普及,人们对生物进化与分类有了更深的理解。
    
\end{abstract}

\newpage
生物进化和系统发育一直是生命科学的一个大论题,这个论题可以归结为“生命从哪里来,生物如何发展成现在这个样子,地球上那么多种生物该如何分类,生物之间又有什么进化关系”。对于这一问题的解释也多种多样,我今天只挑几个有意思的话题展开讨论。\par
我们先聊一聊生物进化。\par 初中课本中写道,生物体的进化方向是从水生到陆生,从简单到复杂,从低等到高等。这一假设虽然可以以化石证据,即在较深的地层中没有找到高等生物的化石,加以佐证,但是一些现存的生物可以反驳这一观点。比如鲸类是一种水生的哺乳动物,盲鱼在发育过程中失去了视觉。这都与传统的理论相反\par 所以科学家们提出了新的理论,现在普遍认可的理论是进化由基因突变的积累造成,是完全随机的过程,自然选择使有利的突变被保留,不利的突变被淘汰,为进化提供方向,同时造就适应,生物就在这样的选择压力下形态发生改变,越来越适应当前的环境。这一理论至今未被推翻,且可以完美解释目前的现象。还以鲸类举例。对基因序列的分析表明,鲸与牛等食草动物亲缘关系较近,同属鲸偶蹄目。可以认为,鲸的祖先是一种小型陆生食草动物,在沧龙灭绝后生态位空闲,碰巧此时古鲸在水边活动,一些适应水生生活的基因(性状)被保留下来,于是这种小型食草动物开始下水,占据了海洋大型捕食者的生态位,逐渐适应水生生活与食肉。\ \ \ 盲鱼长期在洞穴里生活,他们的有眼祖先来到洞穴时,产生多种突变,其中,在漆黑的洞穴中生活,视力减弱的个体更节省能量,且不会对其存活造成威胁,所以视力减弱的个体有更长的生存时间以让他们产生后代,造成无眼基因频率逐渐扩大,加上洞穴的地理隔离,洞穴鱼无法与其他洞穴外的个体进行基因交流,最后,它们的眼睛退化,成了盲鱼。\par
从这两个例子可以看出,生物对环境的适应并非由神灵设计,也不是生物主动变化去适应环境,而是生物随机变化,环境对生物进行选择,足够幸运的个体能够存活,造就进化与适应。\par
说到生物进化,随之而来的一个问题就是,人类是否影响力生物的进化,人类是否可以主观的决定进化的方向,人是否可以进化自己,把人变成“超”人\par
先说人类对其他生物进化的影响。这个答案是:是的,而且影响不小,非洲的非洲象在草原上几乎没有天敌,在性选择的作用下,象牙变得越来越长。但近几十年来,人类盗猎者为了获取象牙,大量猎杀有长牙的象,使得相关基因的频率显著降低,或者说,无长牙的象更容易存活(留下后代)。于是近几年,非洲象种群的象牙长度明显减小。人类修建的大型城市也使得很多生物性状改变,适应城市温暖,食物多的环境。所以人作为生态系统的一个环节,且人个体大,种群数量多,对环境的影响是不可避免的,必然造就生物的进化。\par
人也可以主动的,有意识的决定进化的方向,比较有名的一个例子就是鱼缸里的金鱼。金鱼的祖先是鲫鱼,个体偏大,体色以灰色\ 青色为主,人类为了娱乐或观赏,主动选择有鲜艳体色的个体,看到一些奇怪的突变体(如水泡眼)也会因其造型奇特而加以保留。于是,这些鱼体型变小,体色多变,形态各异,成为了观赏鱼。在自然界中,这些突变多半活不下来,人主动/主观定下了选择的方向,才让鱼向着“残次”的方向进化。\par
关于人主观进化自己这个话题,我的看法是这样的。虽然说,人为了使自己进化,主动定向改变环境,促进自然选择,让我们“自相残杀”这种事几乎不可能发生(伦理上绝对不允许),但随着基因治疗与基因编辑的普及与技术的进步(如现在的CRISPR-CAS9已经实现了少量碱基的精准替换),从理论上来讲,人完全可以改造自己的基因,让人类自己活得更长,疾病更少,或是像蜘蛛侠一样拥有超能力。对于成人来说,这种改造十分困难,因为要对大量个体的几乎所有细胞导入一套外源基因,(成功率小,副作用大),只能做到改造一些干细胞,且类似技术已开始用于治疗癌症(CAR-T细胞),国外从90年代起也有对基因病的基因治疗的研究。但是,人类可以对胚胎干细胞或受精卵进行操作,使婴儿带上外源基因。此前已有对婴儿进行基因编辑的案例。但这种“超"进化极易造成伦理问题(如种族歧视),所以即使未来技术普及了,可能性也不大,或者说,只能用于治疗性用途,无法用于生殖性用途。 然而,人类可以用基因编辑改造动物,植物的基因组,产生新的突变与能达到人目的的品种,虽然谈不上进化,但有着重大的实用意义。就像南大一位教授正在研究的人源实验鼠,把人的免疫基因导入老鼠的基因组中,并敲掉鼠的相关序列,在小白鼠作为实验动物“会得病,症状易观察”的基础上进一步加上免疫系统与人类相同这一优点,可以进一步研究疾病的发病,治疗机理,推动相关研究的发展。\par
然后谈一谈生物分类。\par
传统的观点认为,生物的分类以其形态结构为依据。如苹果,梨因其结构相似被划为同一属。但随着人们研究的深入,发现若选择压力相似,亲缘关系很远的生物也会进化出相同的对策。所以现在人们开始以DNA的关键序列作为分类依据。如最新的被子植物分类系统APG\ IV,用基因序列信息对植物的进化树进行优化,新增了 Boraginales, Dilleniales, Icacinales and Metteniusales几个目,并调整了玫瑰属,苹果属等的发育关系;犹如蝙蝠类最早的分类是以形态分类,分为大蝙蝠亚目(仅一科),小蝙蝠亚目。但近几年对一个关键序列的研究使翼手目进行了大洗牌,重新分为阴翼手亚目,阳翼手亚目,但这两个类群几乎没有明显的形态学差异,估计是由于蝙蝠活动范围广,易产生地理隔离,隔离后趋异进化(造成)。所以生物的分类在书上不是永恒不变的。随着分子生物学的发展,分类系统会不断优化。\par
再来谈一谈人类的系统的发育。很多人都认为人是由猴子进化而来的。其实不然。可以说,人与猿有共同祖先,但千万年前,一部分“祖先”离开了树木,开始适应草原地栖生活,然后开始适应二足行走与群居生活,成了现代智人,也就是我们,另一部分祖先猿留在树上,形成今天的黑猩猩,大猩猩。并不是人们想象的“猴子一代代生孩子,孩子从猿变成人”。并且,人在进化过程中也并非“一枝独秀”,历史上曾经出现过很多人类分支,如能人,直立人,尼人,且这些化石人类并非先后关系,而是像现存生物一样,呈并列关系。但这些史前人类相继灭绝,只有智人存活到了今天。并适应了现在的环境。



\end{document}